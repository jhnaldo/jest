\section{Related Work}\label{sec:related}

Our technique is related to two research fields: differential testing, fuzzing,
and fault localization.

\textbf{Differential Testing:} Differential testing\cite{???} utilizes multiple
implementations as cross-referencing oracles to find semantics bugs.
Researchers applied this technique for various applications, such as Java
Virtual Machine (JVM) implementations\cite{???}, web applications\cite{???},
and binary lifters\cite{ir-diff-test}.  Moreover, \textsc{Nezha} introduces a
guided differential testing tool with the concept of $\delta$-diversity to
efficiently find semantic bugs.  However, they have fundamental limitations because
they use only the cross-referencing oracles and target potential bugs in
implementations.  Our $N$+1-differential testing extends the idea of
differential testing with not only $N$ different implementations but also a
mechanized specification to test both of them.  Moreover, our approach
automatically generates conformance tests directly from the specification.

\textbf{Fuzzing:} Fuzzing is a software testing technique for detecting security
vulnerabilities by generating or mutating test inputs.  For JavaScript engines,
Han et al.\cite{codealchemist} presented CodeAlchemist that generates JavaScript
code snippets based on semantics-aware assembly, Wang et al.\cite{???}
presented Superion using Grammar-aware greybox fuzzing, and Park et
al.\cite{???} presented \textsc{Die} using an aspect-preserving mutation,.
However, they only focus on finding security vulnerabilities but not the
semantics correctness. Unlike fuzzing approaches, our $N$+1-differential testing
focuses on semantics bugs by comparing the mechanized specification with
multiple implementations.  Moreover, it could localize not only specification
bugs in ECMAScript but also bugs in JavaScript engines indirectly using the bug
locations in ECMAScript.

% CodeAlchemist: Semantics-Aware Code Generation to Find Vulnerabilities in JavaScript Engines.
% Fuzzing JavaScript Engines with Aspect-preserving Mutation
% Superion: grammar-aware greybox fuzzing.

\textbf{Fault Localization:} To localize detected bugs in ECMAScript, we used
Spectrum Based Fault Localization (SBFL)\cite{???}, which is a ranking technique
based on likelihood of being faulty for each program element.
Tarantula\cite{???, ???} was the first tool that supports SBFL with a simple
formula and many formulas have been developed\cite{???} to increase the accuracy
of bug localization.  Moreover, Sohn et al.\cite{???} introduced a novel
approach for fault localization technique using code and change metrics via
learning of SBFL formulas.  While we utilize the basic formula introduced by
Tarantula, we believe that it is possible to improve the accuracy of bug
localization by using more advanced SBFL techniques.

% Visualization of test information to assist fault localization.
% Visualization for Fault Localization
% FLUCCS: Using Code and Change Metrics to Improve Fault Localization

% \begin{itemize}
%   \item CodeAlchemist\cite{codealchemist}
%   \item Csmith\cite{csmith}
%   \item Grammar-based Whitebox Fuzzing\cite{grammar-whitebox}
%   \item Montage\cite{montage}
%   \item QuickCheck\cite{quickcheck}
%   \item SAGE\cite{sage}
%   \item Rnadom String from a CFG\cite{cfg-gen}
%   \item differential Testing for Lifter\cite{ir-diff-test}
%   \item NEZHA\cite{nezha}
%   \item JavaScript History\cite{js-hopl}
%   \item JISET\cite{jiset}
% \end{itemize}
