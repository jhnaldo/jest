\begin{abstract}
  Modern programming follows the continuous integration (CI) and continuous
  deployment (CD) approach rather than the traditional waterfall model.  Even
  the development of modern programming languages uses the CI/CD approach to
  swiftly provide new language features and to adapt to new development
  environments.  Unlike in the conventional approach, in the modern CI/CD
  approach, a language specification is no more the Oracle of the language semantics
  because both the specification and its implementations (interpreters or compilers) can co-evolve.
  In this setting, both the specification and implementations may have bugs, and
  guaranteeing their correctness is non-trivial.

  In this paper, we propose a novel \textit{$N$+1-version differential testing} to resolve
  the problem.  Unlike the traditional differential testing, our approach
  consists of three steps: 1) to automatically synthesize programs guided by the
  syntax and semantics from a given language specification, 2) to generate
  conformance tests by injecting assertions to the synthesized programs to check their final program
  states, and 3) to find and localize bugs in the specification and implementations via executing
  the conformance tests on multiple implementations.  We actualize our approach for
  the JavaScript programming language via \( \tool \), which performs
  $N$+1-version differential testing for modern JavaScript engines and ECMAScript,
  the language specification describing the syntax and semantics
  of JavaScript in a natural language.  We evaluated \( \tool \) with four JavaScript engines that
  support all modern JavaScript language features and the latest version of
  ECMAScript (ES11, 2020).  \( \tool \) automatically synthesized 1,700
  programs that covered 97.78\% of syntax and 87.70\% of
  semantics from ES11.  Using the assertion-injected JavaScript programs,
  it detected 44 engine bugs in four different engines and 27
  specification bugs in ES11.
\end{abstract}
