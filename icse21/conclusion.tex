\section{Conclusion}\label{sec:conclude}
The development of modern languages follow continuous integration (CI) and
continuous deployment (CD) approach to instantly support fast changing user
demands.  It make more difficult to find both of semantics bug in implementations
and the specification.  To alleviate this problem, we introduce
$N$+1-version differential testing, which is the first technique to test both of
implementations and the specification.  We actualized our approach for the
JavaScript programming language via $\tool$, which performs $N$+1-version differential
testing for modern JavaScript engines and their specification, ECMAScript.  It
automatically generated \inred{-} JavaScript programs with \inred{-}\% of syntax
coverage and \inred{-}\% of semantics coverage on the latest version of
ECMAScript (ES11, 2020).  Our tool injected assertions to them in order to
convert them as conformance tests.  We executed generated conformance tests on
four different modern JavaScript engines: GraalJS, QuickJS, Moddable XS, and V8.
Using the execution results, we found \inred{-} specification bugs and
\inred{-} engine bugs (\inred{-} for GraalJS, \inred{-} for QuickJS, \inred{-}
for Moddable XS, and \inred{-} for V8). All of them confirmed by the TC39, the
committee of ECMAScript, and authors of JavaScript engines and will be fixed.
We also localized detected bugs on ECMAScript and the average ranks of actual
bug locations are \inred{-} for algorithms and \inred{-} for algorithm steps.
