\section{Conclusion}\label{sec:conclude}
The development of modern programming languages follows the continuous integration (CI) and
continuous deployment (CD) approach to instantly support fast changing user demands.
Such continuous development makes it difficult to find semantics bugs
in both the language specification and its various implementations.
To alleviate this problem, we present $N$+1-version differential testing,
which is the first technique to test both implementations and its specification in tandem.
We actualized our approach for the JavaScript programming language via $\tool$,
using four modern JavaScript engines and the latest version of ECMAScript (ES11, 2020).
It automatically generated \inred{X} JavaScript programs with \inred{X}\% of syntax
coverage and \inred{X}\% of semantics coverage on ES11.  $\tool$ injected assertions
to the generated JavaScript programs to convert them as conformance tests.
We executed generated conformance tests on four engines that support ES11:
GraalJS, QuickJS, Moddable XS, and V8.  Using the execution results,
we found \inred{X} specification bugs and \inred{X} engine bugs (\inred{X} for GraalJS,
\inred{X} for QuickJS, \inred{X} for Moddable XS, and \inred{X} for V8).
All the bugs were confirmed by TC39, the committee of ECMAScript, and
the corresponding engine teams, and they will be fixed in the specification and the engines.
We believe that $\tool$ takes the first step towards co-evolution of
software specifications, tests, and their implementations for CI/CD.
