\subsection{Bug Localizer}

In the bug detection/localization phase, \textsf{Bug Localizer} detects and
localizes bugs based on the execution results of given conformance tests on
multiple JavaScript engines.  If most of engines fails to a specific
conformance test, we believe that the failure of the test represents the
correct semantics.  Thus, we suspect engines passing the test and the
specification of containing bugs related to the test.  Otherwise, we assume
that bugs exist in engines failing the test.

After targeting a poterntially buggy engine or specification with failed tests,
we utilize Spectrum Based Fault Localization (SBFL)\cite{sbfl-survey}, which is
which is a ranking technique based on likelihood of being faulty for each
program element.  \inred{We use the following basic formula introduced by Jones
et al. with a tool Tarantula\cite{tarantula, tarantula2}}:
\[
  {{n_{ef}}\over{n_{ef} + n_{nf}}} /
  \left(
    {{n_{ef}}\over{n_{ef} + n_{nf}}} +
    {{n_{ef}}\over{n_{ep} + n_{np}}}
  \right)
\]
where $n_{ef}$, $n_{ep}$ , $n_{ep}$, and $n_{np}$ represent the number of test
cases; subscripts ${}_e$ and ${}_n$ represent whether the test case touched the
relevant program elements or not, and subscripts ${}_f$ and ${}_p$ represent
whether the test case is failed or passed.

We classify the failed test cases based on their failure messages and apply SBFL
with the above formula to each failure case.  To enhance the accuracy of the bug
localization, we expand the set of failed test cases.  We utilize
\textsf{Program Mutator} to get a new program but still failed with the same
message.  We use two different program elements as targets of SBFL: abstract
algorithms and their steps.  While we directly apply the above formula to
calculate the SBFL scores for algorithm steps, we use method level aggregation
introduced by Sohn et al.\cite{fluccs} for abstract algorithms.  in other
words, we aggregate SBFL scores for steps up to algorithm level based on the
highest score among those from steps of each algorithm.
